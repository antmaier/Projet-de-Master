In the general context of cell biology, the study of protein folding is an important topic since their natural tendency to fold into their native conformation is essential for their biological function. This intrinsically stochastic process can lead to misfolding and resulting in amyloids, aggregates of misfolded proteins, which are associated with numerous diseases such as Alzheimer's disease, Parkinson's disease, and type 2 diabetes\cite{chiti_protein_2017}. To prevent this, cells have developed a complex machinery of molecular chaperones to assist the folding of other proteins. Understanding their mechanism of action is essential as it could lead to new therapeutic strategies to treat amyloid-related diseases.

This work focuses on a particular class of molecular chaperones, Hsp100 disaggregases, which belong to the AAA+ ATPase superfamily\cite{tucker_aaa+_2007}. These proteins are typically hexameric oligomers with a central channel through which they translocate substrates\cite{mayer_gymnastics_2010}. This translocation pulls and unfolds misfolded polypeptides out of aggregates that are then dissolved or refolded with the help of other chaperones. However, the mechanism of translocation is still poorly understood, and it is the subject of this work. Shorter J. et al. have proposed a model of the translocation mechanism based on structural biology data, in which a sequential hand-over-hand motion of the protomers is responsible for the translocation\cite{shorter_spiraling_2019}. We will refer to this model as the Sequential Clockwise/2-Residue Step (SC/2R) model. In this work, we propose an alternative model, the Random Protomer Concertina Locomotion (RPCL) model, based on simple physical assumptions about the interaction potential between two neighboring protomers and the resulting minimal energy configuration of the ATPase. We translate these models into the kinetic scheme framework to study the dynamics of the ATPase via simulations and analytical calculations. By comparing the two models with experimental data, we ultimately argue that the RPCL model should be favored over the SC/2R model. The method used is general and could be applied to other translocation models, although this work focuses on the SC/2R and RPCL models.

This document is organized as follows. In Sec.~\ref{sec:theory}, we introduce the kinetic scheme theory, the formulas to compute the quantities of interest, and the simulation algorithm. In Sec.~\ref{sec:models}, we detail the SC/2R and RPCL models and translate them into the kinetic scheme framework. In Sec.~\ref{sec:experiments}, we present some experiments we conducted on the dynamics of the two models: a direct comparison in ideal conditions, the effect of an external force, and the effect of a defective protomer. Finally, in Sec.~\ref{sec:conclusion}, we conclude and discuss some perspectives for future research.
