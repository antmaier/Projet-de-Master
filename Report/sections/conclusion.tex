In this work, we have studied two models of the mechanism of action of AAA+ ATPase, the Sequential Clockwise/2-Residue Step (SC/2R) model based on structural biology data and its analysis made by Shorter J. et al.\cite{shorter_spiraling_2019}, and our alternative model, the Random Protomer Concertina Locomotion (RPCL) model based on physical assumptions about the nature of the interaction potential between two neighboring protomers and the resulting minimal energy configuration of the ATPase, which are a flat and a staircase-like configurations, both having been observed experimentally\cite{shorter_spiraling_2019}. To conduct our study, we have implemented the two models in the kinetic scheme framework, which was first introduced in detail. The idealized models are both made of a single loop. Still, the key difference resides in the fact that in the SC/2R model, each protomer consecutively undergoes a hydrolysis-translocation-exchange reaction sequence. In contrast, in the RPCL model, all protomers compete in parallel to hydrolyze their ATP molecule, and the first one to do so induces the substrate translocation. Using simulations and analytical calculations, we analyzed the dynamics of the ATPase from the perspective of both models, focusing on the average velocity and the ATP consumption rate.

Comparing sample trajectories of the two models, we have shown that at equal average velocity, RPCL consumes five times less ATP and has a translocation steplength five times longer than SC/2R by moving steps five times longer, which better matches experimental data\cite{avellaneda_processive_2020}. We have also studied the response of the ATPase to an external pulling force opposing the substrate translocation. We have shown that for both models it results in a sigmoidal shape of the average velocity inversely related to the force, which is can be explained using the analytical solution of the kinetic scheme, but was not the expected result looking at experimental data which instead predicts a threshold force intensity under which the velocity is not significantly affected and over which the velocity drastically falls. However, the measurements were subject to high uncertainties and should be investigated further\cite{avellaneda_processive_2020}. Finally, we have studied the effect of a defective protomer on the dynamics of the ATPase, and we have shown that when one protomer has a reduced hydrolysis rate, the average velocity is visibly reduced for the SC/2R model, whereas it is almost unaffected for the RPCL model, which is consistent with experimental data\cite{desantis_operational_2012}.

In light of these results, we argue that the RPCL model should be favored over the SC/2R model, as its simulated trajectories and dynamics better reproduce experimental data. The predicted existence of a flat configuration, the larger translocation stepsize and the behavior in case of a defective protomer are three key points that enable us to discriminate the two models. 

We believe that the RPCL model is a good starting point to understand the dynamics of the ATPase, but it is still a very idealized model, raising numerous questions. Here we list some of them:
\begin{itemize}
    \item What is the nature of the interaction potential between two neighboring protomers, and is the linear approximation appropriate? If not, does the new potential energy landscape results in the same minimal energy configurations?
    \item How is post-hydrolysis protomer modified on a structural level to alter the interaction potential with its neighbor?
    \item Why are some protomers fixed to the substrate during the extension and contraction while some others are not?
    \item Are there other configurations observed experimentally, and how can we extend the RPCL model to encompass them?
    \item The extension and contraction are modeled as occurring abruptly after a sojourn time sampled from an exponential distribution. What is the physics underlying these reactions, and is this simplification appropriate?
    \item This idealized model only considers states belonging to the main loop, where at most one protomer is in ADP-bound state. We only skimmed over the possibility of leaving this main loop by adding a single \emph{out} state reachable from any state. Should one consider a more complex model with more states?
\end{itemize}

Science is an iterative process where empirical, numerical, and theoretical advances are poetically complementary, and this work paved the way for future research on the dynamics of AAA+ ATPases. Moving forward, the next step would be to compare the RPCL model with various experimental setups in order to refine the model or even discard it in favor of a better one. Other alternative models have been proposed\cite{lin_aaa+_2022}, and it would be interesting to translate them into the kinetic scheme framework to see how they compare with the RPCL model. Experiments would also help to fine-tune all the degrees of freedom of the model, namely all the rates; even though the qualitative results are not expected to change, we should not be too confident; interesting physics can lie where we least expect it. 
